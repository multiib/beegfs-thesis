\chapter{Introduction}

\section{Motivation}
In an era where scientific discovery and technological innovation are accelerating at unprecedented rates, high-performance computing (HPC) has become the backbone of global advancement. From simulating complex molecular interactions to modeling climate change, HPC systems allow us to address challenges once thought impossible. As data demands grow, even the smallest improvements in these systems can make a difference. Integrating Remote Direct Memory Access (RDMA) with BeeGFS over PCI Express could offer one such subtle push, where every gain—no matter how incremental—helps optimize performance and sustain the momentum of progress.
\section{Problem Statement}


HPC environments require storage solutions that can handle large amounts of data with minimal delay and maximum speed. BeeGFS™, developed by ThinkParQ, is a global shared parallel filesystem widely used in HPC for its scalability and high performance. While BeeGFS can use different network setups with TCP/IP, it relies on RDMA only when connected through InfiniBand networks. RDMA improves data transfer speeds and reduces delays by allowing computers to directly access each other's memory, bypassing the operating system and the CPU.
\\\\
However, when BeeGFS operates over PCI Express (PCIe) Non-Transparent Bridge (NTB) connections provided by Dolphin Interconnect Solutions, it cannot use RDMA and instead relies on standard TCP/IP protocols. This limitation may increase delay and reduce overall performance, impacting data-intensive HPC applications that rely on fast data access and transfers.
\\\\
Dolphin Interconnect Solutions offers SuperSockets, a technology that enables RDMA over PCIe NTB. Integrating SuperSockets into BeeGFS could enable RDMA over PCIe NTB, reducing delay and boosting bandwidth compared to TCP/IP. 
\\\\

The goal of this thesis is to integrate SuperSockets™ into BeeGFS to enable RDMA over Dolphin’s PCIe NTB Fabric. This integration aims to reduce latency and increase bandwidth, improving BeeGFS performance in HPC environments that use PCIe NTB networks. The work will involve modifying the BeeGFS source code to support SuperSockets and conducting a performance evaluation. This evaluation will compare BeeGFS with RDMA over PCIe NTB against the existing RDMA setup over InfiniBand and the current PCIe setup over TCP/IP. Key metrics for comparison will include latency, bandwidth, and overall system performance.

\section{Research Objectives}
The main objective of this thesis is to integrate PCIe RDMA functionality into BeeGFS. This integration will use PCIe NTBs from Dolphin Interconnect Solutions to reduce latency, increase bandwidth, and improve reliability in HPC environments.

\subsection{Specific Objectives}
\begin{itemize}
    \item \textbf{Integration of SuperSockets into BeeGFS}: Modify the BeeGFS codebase to support SuperSockets, enabling RDMA over PCIe NTB networks.
    \item \textbf{Performance Evaluation}: Assess the impact of SuperSockets integration on BeeGFS performance metrics such as latency, bandwidth, and throughput.
    \item \textbf{Comparative Analysis}: Compare the performance of BeeGFS over PCIe NTB with SuperSockets against its performance over InfiniBand networks with existing RDMA support.
    \item \textbf{Feasibility Study}: Analyze the practicality and benefits of extending RDMA support in BeeGFS to PCIe NTB networks for broader applicability in HPC infrastructures.
\end{itemize}

\subsection{Research Questions}
\begin{enumerate}
    \item \textit{How can SuperSockets be effectively integrated into the BeeGFS filesystem to enable RDMA over PCIe NTB networks?}
    \item \textit{What are the performance improvements in terms of latency and bandwidth when using BeeGFS with SuperSockets over PCIe NTB compared to standard TCP/IP communication?}
    \item \textit{How does the performance of BeeGFS with SuperSockets over PCIe NTB compare to its performance with RDMA over InfiniBand networks?}
    \item \textit{What challenges may arise during the integration of SuperSockets into BeeGFS, and how can they be addressed?}

\end{enumerate}

\section{Scope and Limitations}
\subsection{Scope}
This thesis focuses on the technical integration and performance evaluation of SuperSockets within the BeeGFS filesystem over PCIe NTB networks. The study encompasses:
\begin{itemize}
    \item \textbf{Software Development}: Modifying the BeeGFS codebase to incorporate SuperSockets support.
    \item \textbf{Experimental Setup}: Configuring an HPC environment/POC using Dolphin's PCIe NTB interconnects for testing and benchmarking.
    \item \textbf{Performance Metrics}: Measuring and analyzing key performance indicators such as latency, bandwidth, and system throughput.
    \item \textbf{Comparative Analysis}: Evaluating the enhanced BeeGFS over PCIe NTB against its existing RDMA implementation over InfiniBand networks.
.
\end{itemize}

\subsection{Limitations}

\begin{itemize}
    \item \textbf{Hardware Constraints}: The experimental results are limited to the specific hardware and network configurations available during the study, which may not represent all possible HPC environments.
    \item \textbf{Generality of Findings}: While the findings aim to be indicative, they may not be directly generalizable to all versions of BeeGFS or different RDMA technologies.
    \item \textbf{Time Constraints}: The scope of the thesis is confined to the allocated time frame, which may limit the depth of exploration into advanced optimizations or extended testing scenarios.
    \item \textbf{Focus on SuperSockets (?)}: The study specifically investigates SuperSockets integration and does not explore alternative methods of enabling RDMA over PCIe NTB networks. ! May use SISCI here..
    \item \textbf{Software Stability}: Potential stability issues arising from modifications to the BeeGFS codebase may affect performance results and require additional troubleshooting beyond the thesis scope.
\end{itemize}
