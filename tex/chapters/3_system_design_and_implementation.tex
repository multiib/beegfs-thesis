\chapter{System Design and Implementation}

This chapter outlines the technical design choices and the integration of SuperSockets into the BeeGFS codebase. It begins with a design overview and a review of the existing logic in the source code, followed by implementation details and validation. Finally is a summary of the scripting logic used in the work.

\section{Design Overview}

SuperSockets are designed for plug-and-play integration, requiring minimal changes to existing systems. The implementation follows the standard socket logic flow to ensure compatibility.

The client kernel module operates in kernel space, so kernel-space SuperSockets must be used. For other components, both userspace and kernel-space implementations are provided.

In the userspace implementation, application code remains unmodified. Interception of socket calls is achieved using \texttt{LD\_PRELOAD}.

\section{Existing Logic}

BeeGFS has, as described in the background chapter, multiple services or components. The BeeGFS source tree is structured by service, with every component residing in a separate directory and building independently. It also includes a \texttt{common} directory that holds shared code used across components. While most of the codebase is written in C++, the client kernel module is implemented in C. Socket logic is shared mainly between components.

Sockets play a central role in BeeGFS, enabling clients to connect to metadata servers to locate files and to storage servers to read or write file chunks. Metadata and storage servers also communicate with each other to ensure data consistency.
BeeGFS uses TCP for reliable, stream-based file transfers and UDP for less critical messages like heartbeat messages to the management service. To support many simultaneous connections, BeeGFS uses connection pooling, which reduces the overhead of creating new connections for frequent operations like metadata updates or data streaming.

The system selects the best network interface for each connection, whether Ethernet or InfiniBand.

BeeGFS abstracts low-level socket operations through a layered class structure. At the core is the \texttt{Socket} class, which handles basic network actions like connecting, binding, and formatting addresses. Building on this, the \texttt{PooledSocket} class supports connection pooling, timeouts, and connection reuse—key for managing resources efficiently in large-scale environments.

The \texttt{StandardSocket} class extends this further by implementing full socket behavior, including \texttt{epoll}-based non-blocking I/O, configuration of socket options, and support for both TCP and UDP. These sockets are used for data transfers, metadata queries, and internal communication between BeeGFS components.

BeeGFS also scans available network interfaces using the \texttt{NetworkInterfaceCard} class to support high-performance networking. It discovers standard TCP/IP interfaces and checks for RDMA-capable devices by attempting to bind RDMA sockets to each address. This allows BeeGFS to dynamically choose between standard and high-speed interfaces like InfiniBand, based on availability and configuration.

This modular socket and interface design enables BeeGFS to maintain flexibility, performance, and scalability across different environments and workloads.

BeeGFS uses a \texttt{NodeConnPool} class to manage connections efficiently. Each node maintains a pool of reusable socket connections to other nodes. These are based on the \texttt{PooledSocket} class, which extends basic socket functionality with features like expiration timers, connection reuse, and availability tracking. When a service needs to communicate, it requests a stream socket from the pool, either by reusing an existing connection or establishing a new one.

\texttt{NodeConnPool} automatically selects the best available network interface, supports RDMA and TCP, and applies socket options to optimize performance. It also handles authentication, connection fallback, and clean-up idle or failed sockets. This pooling mechanism reduces overhead, increases scalability, and ensures robust, low-latency communication across the cluster.

\section{Implementation Details}

SuperSockets are designed to be plug-and-play, requiring minimal modifications to the existing networking logic in BeeGFS. The core implementation task involved locating where TCP sockets are defined in the source code and replacing the standard \texttt{AF\_INET} address family with \texttt{AF\_SSOCKS}. This change allows the system to establish socket connections over Dolphin's PCIe-based interconnect rather than through traditional Ethernet. In Listing~\ref{lst:ssocks_construct}, the SuperSocket address family constant is used in the function that creates a standard socket. This implementation resides in \texttt{StandardSocket.c}. The function is called from \texttt{NodeConnPool.c}, where client-side socket management is handled. A similar structure is used in the C++ implementation for the user-space daemons. The full source code is available in the linked GitHub repository.

In the current implementation, the \texttt{AF\_SSOCKS} constant is hardcoded into the codebase. A more robust approach would involve dynamically retrieving the assigned address family value at runtime, which is discussed further in the future work section.

By default, SuperSockets use address family 27. However, this conflicts with \texttt{AF\_SDP\_INET}, which is associated with the Socket Direct Protocol (SDP) used in InfiniBand environments~\cite{goldenberg}. To avoid this conflict, the address family was explicitly set to 34 using Dolphin's configuration mechanism. This value was specified in the \texttt{dis\_ssocks.conf} file.

UDP sockets in the BeeGFS codebase were left unchanged. These sockets are used solely for lightweight heartbeat messages and do not contribute to the performance of data transfer operations. As such, they continue to operate over standard Ethernet interfaces.
 
\begin{listing}[H]
\begin{minted}[label={lst:ssocks_construct}]{c}
StandardSocket* StandardSocket_constructUDP(void) {
   return StandardSocket_construct(PF_INET, SOCK_DGRAM, 0);
}

StandardSocket* StandardSocket_constructTCP(void) {
   return StandardSocket_construct(PF_SSOCKS, SOCK_STREAM, 0);
}

StandardSocket* StandardSocket_constructSDP(void) {
   return StandardSocket_construct(PF_SDP, SOCK_STREAM, 0);
}
\end{minted}
\caption{Kernel Module Socket Construction}
\end{listing}

The \texttt{AF\_SSOCKS} address family is also defined in the \texttt{NetworkInterfaceCard} class. This class is responsible for detecting and managing network interfaces, and for distinguishing between TCP and existing RDMA functionality. It opens a new socket to interact with the network stack.

\section{Validation}

Dolphin Interconnect Solutions provides a utility called \texttt{dis\_ssocks\_stat}, which displays real-time information about active SuperSockets connections on a system. Since SuperSockets automatically falls back to TCP/IP if a connection fails, this tool helps determine whether data is transmitted via SuperSockets or the fallback method. Specifically, it reports the number of bytes sent and received through SuperSockets and fallback TCP/IP. Listing \ref{lst:ssocks_stat} shows an example of terminal output from \texttt{dis\_ssocks\_stat} during a successful SuperSockets file transfer in BeeGFS.

\begin{listing}[H]
\begin{minted}[
  linenos        = false,
  style          = vs,
  bgcolor        = gray!10,
  fontsize       = \small,
  highlightlines = {1},            % which line(s) to colour
  highlightcolor = purple!20,
]{cucumber}
./dis_ssocks_stat -G
STREAM: 20
DGRAM: 0 (1 tx, 1 rx)
RDS: 0
SuperSockets in: 185112645 (176 MiB)
SuperSockets out: 17095 (0 MiB)
Fallback in: 0 (0 MiB)
Fallback out: 0 (0 MiB)
TX:  0.0% poll,  0.0% block
RX:  0.0% poll,  0.0% block
Interrupts TX: 370 req, 357 noreq, 0 task
Interrupts RX: 0 req, 6094 noreq, 0 task
Inline: 0 crc, 0 resend, 0 discard
\end{minted}
\caption{Terminal output of SuperSockets statistics}
\label{lst:ssocks_stat}
\end{listing}

\section{Scripts}

Several scripts were developed throughout this work to facilitate various tasks related to software development, performance testing, and data analysis. These scripts, available in the attached repository, automated repetitive processes, reduced manual effort, and ensured consistency across different stages of experimentation.

The \texttt{cp-binaries} script automates the transfer of compiled binary files to their designated directories. Streamlining this process ensures that the most recent versions of binaries are consistently available for testing, minimizing the potential for errors caused by manual copying.

The \texttt{fio-run} script manages the execution of performance benchmarks using the Flexible I/O Tester (FIO). It simplifies and standardizes the benchmarking process, facilitating consistent, reproducible evaluations of storage performance under various configurations and workloads.

The \texttt{enable-if} script simplifies the enabling and disabling of network interfaces. This functionality is useful when switching between interfaces during benchmarking. It works by manipulating the \texttt{connInterfacesFile}.

The \texttt{beegfs-swap} script facilitates convenient switching between different BeeGFS builds, streamlining debugging, testing, and benchmarking workflows. It provides options to swap individual BeeGFS services or all services simultaneously. Additionally, the script supports comparing the currently deployed build with the version under development.

The \texttt{beegfs-service} script provides capabilities for managing BeeGFS services. Since BeeGFS services must be initiated and terminated in a specific sequence, this script ensures the correct ordering. Additionally, it saves time by allowing all services to be started or stopped simultaneously.

The \texttt{sync} script handles synchronization of files between computers with the help of \texttt{rsync}.

The \texttt{div} script supports dividing large datasets or log files into smaller, manageable segments. This capability is particularly beneficial for parallel processing, detailed analysis, and iterative testing processes.

The series of \texttt{make\_img\_*} scripts serve as configuration tools for the Matplotlib-based Python plotting scripts. They are designed to generate multiple plots efficiently, with the flexibility to adjust plot settings for individual plots easily.
